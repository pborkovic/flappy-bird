% ============================================
% Weather System
% ============================================

\chapter{Weather System}

\section{Overview}

Dynamic weather and time-of-day system rendered via \texttt{CanvasLayer} (layer 10).

\section{Enumerations}

\subsection{WeatherType}

\begin{lstlisting}[language={[Sharp]C}]
public enum WeatherType
{
    Clear,  // No weather effects
    Rain,   // Particle-based rain
    Fog     // Screen overlay
}
\end{lstlisting}

\subsection{TimeOfDay}

\begin{lstlisting}[language={[Sharp]C}]
public enum TimeOfDay
{
    Day,    // No overlay
    Night   // Dark tint overlay
}
\end{lstlisting}

\section{WeatherController}

\subsection{Configuration Constants}

\begin{table}[h]
\centering
\begin{tabular}{|l|l|l|}
\hline
\textbf{Parameter} & \textbf{Value} & \textbf{Description} \\
\hline
WeatherChangeInterval & 30.0s & Auto-change period \\
TransitionDuration & 2.0s & Effect fade duration \\
\hline
\end{tabular}
\caption{WeatherController configuration}
\end{table}

\subsection{Visual Effects}

\begin{table}[h]
\centering
\begin{tabular}{|l|l|l|}
\hline
\textbf{Effect} & \textbf{Implementation} & \textbf{Properties} \\
\hline
Night & ColorRect overlay & Color(0, 0, 0.15, 0.6) \\
Rain & GpuParticles2D & 300 particles, diagonal fall \\
Fog & ColorRect overlay & Color(0.85, 0.85, 0.9, 0.5) \\
\hline
\end{tabular}
\caption{Weather visual effects}
\end{table}

\subsection{Public API}

\begin{lstlisting}[language={[Sharp]C}]
// Set specific weather/time
void SetWeather(WeatherType weather)
void SetTimeOfDay(TimeOfDay timeOfDay)

// Randomize both
void RandomizeWeather()

// Control auto-cycling
void SetAutoChangeWeather(bool enabled)

// Reset to Clear/Day
void ResetWeather()

// Getters
WeatherType GetCurrentWeather()
TimeOfDay GetCurrentTimeOfDay()
\end{lstlisting}

\section{Scene Structure}

\begin{lstlisting}
WeatherSystem (CanvasLayer, layer=10)
+-- WeatherController (Node)
+-- NightOverlay (ColorRect)
|   +-- mouse_filter = IGNORE
|   +-- anchors = full screen
+-- FogOverlay (ColorRect)
|   +-- mouse_filter = IGNORE
|   +-- anchors = full screen
+-- RainParticles (GpuParticles2D)
    +-- amount = 300
    +-- position = (960, -50)
\end{lstlisting}

\section{Rain Particle Configuration}

\begin{table}[h]
\centering
\begin{tabular}{|l|l|}
\hline
\textbf{Property} & \textbf{Value} \\
\hline
Direction & (0.2, 1, 0) \\
Spread & 8.0\textdegree \\
Initial velocity & 600-800 \\
Gravity & (50, 400, 0) \\
Scale range & 0.8-1.2 \\
Blend mode & Additive \\
\hline
\end{tabular}
\caption{Rain particle material settings}
\end{table}

\section{Transition System}

Effects use Godot's \texttt{Tween} for smooth transitions:

\begin{lstlisting}[language={[Sharp]C}]
_currentTween?.Kill();
_currentTween = CreateTween();
_currentTween.TweenProperty(
    _nightOverlay,
    "color:a",
    0.6f,
    TransitionDuration
);
\end{lstlisting}

\section{Input Handling}

All overlay \texttt{ColorRect} nodes have \texttt{mouse\_filter = 2} (IGNORE) to prevent blocking UI interactions.
